\section{Introduction}


\subsection{}
To minimise 
%
\begin{equation}
E(\textbf{w}) = \frac{1}{2} \sum_{n = 1}^{N} \left( y(x_n, \textbf{w}) - t_n \right) ^ 2,
\end{equation}
%
setting its derivative as zero gives
%
\begin{equation}
\textbf{0} = \sum_{n = 1}^{N} \frac{\partial y(x_n, \textbf{w})}{\partial \textbf{w}} \left( y(x_n, \textbf{w}) - t_n \right).
\end{equation}
%
Substituting 
%
\begin{equation}
y(x_n, \textbf{w}) = \sum_{j = 0}^{M} w_j x_n^j
\end{equation}
%
gives
%
\begin{equation}
0 = \sum_{n = 1}^{N} x_n^i \left( \sum_{j = 0}^{M} w_j x_n^j - t_n \right).
\end{equation}
%
Therefore, we have 
%
\begin{equation}
\sum_{j = 0}^{M} A_{ij} w_j = T_i
\end{equation}
%
where
%
\begin{equation}
\begin{aligned}
A_{ij} &= \sum_{n = 1}^{N} x_n^{i + j}, \\
T_i &= \sum_{n = 1}^{N} x_n^i t_n.
\end{aligned}
\end{equation}


\subsection{}
To minimise 
%
\begin{equation}
\tilde{E}(\textbf{w}) = \frac{1}{2} \sum_{n = 1}^{N} \left( y(x_n, \textbf{w}) - t_n \right) ^ 2 + \frac{\lambda}{2} \lVert \textbf{w} \rVert ^ 2,
\end{equation}
%
setting its derivative as zero gives
%
\begin{equation}
\textbf{0} = \sum_{n = 1}^{N} \frac{\partial y(x_n, \textbf{w})}{\partial \textbf{w}} \left( y(x_n, \textbf{w}) - t_n \right) + \lambda \textbf{w}.
\end{equation}
%
Substituting 
%
\begin{equation}
y(x_n, \textbf{w}) = \sum_{j = 0}^{M} w_j x_n^j
\end{equation}
%
gives
%
\begin{equation}
0 = \sum_{n = 1}^{N} x_n^i \left( \sum_{j = 0}^{M} w_j x_n^j - t_n \right) + \lambda w_i.
\end{equation}
%
Therefore, we have 
%
\begin{equation}
\sum_{j = 0}^{M} \tilde{A}_{ij} w_j = T_i
\end{equation}
%
where
%
\begin{equation}
\begin{aligned}
\tilde{A}_{ij} &= \sum_{n = 1}^{N} x_n^{i + j} + \lambda \delta_{ij}, \\
T_i &= \sum_{n = 1}^{N} x_n^i t_n.
\end{aligned}
\end{equation}


\subsection{}
Let $a$, $o$ and $l$ be the events where an apple, orange and lime are selected respectively.
The probability that an apple is selected is given by
%
\begin{equation}
p(a) = p(a | r) p(r) + p(a | b) p(b) + p(a | g) p(g).
\end{equation}
%
Substituting $p(a | r) = \frac{3}{10}$, $p(r) = \frac{1}{5}$, $p(a | g) = \frac{1}{2}$, $p(r) = \frac{1}{5}$, $p(a | g) = \frac{3}{10}$ and $p(g) = \frac{3}{5}$ gives
%
\begin{equation}
p(a) = \frac{17}{50}.
\end{equation}
%

If an orange is selected, the probability that it came from the geen box is given by
%
\begin{equation}
p(g | o) = \frac{p(g, o)}{p(o)}.
\end{equation}
%
Here,
%
\begin{equation}
\begin{aligned}
p(g, o) &= p(o | g) p(g), \\
p(o) & = p(o | r) p(r) + p(o | b) p(b) + p(o | g) p(g).
\end{aligned}
\end{equation}
%
Substituting $p(o | r) = \frac{2}{5}$, $p(r) = \frac{1}{5}$, $p(o | b) = \frac{1}{2}$, $p(b) = \frac{1}{5}$, $p(o | g) = \frac{3}{10}$ and $p(g) = \frac{3}{5}$ gives $p(g, o) = \frac{9}{50}$ and $p(o) = \frac{9}{25}$.
%
Therefore,
\begin{equation}
p(g | o) = \frac{1}{2}.
\end{equation}


\stepcounter{subsection}


\subsection{}
By the definition, 
%
\begin{equation}
{\rm var} f(x) = {\rm E} \left( f(x) - {\rm E} f(x) \right) ^ 2.
\end{equation}
%
The right hand side can be written as 
%
\begin{equation}
{\rm E} \left( \left( f(x) \right) ^ 2 - 2 f(x) {\rm E} f(x) + \left( {\rm E} f(x) \right) ^ 2 \right) = {\rm E} \left( f(x) \right) ^ 2 - \left( {\rm E} f(x) \right) ^ 2.
\end{equation}
%
Therefore, 
%
\begin{equation}
{\rm var} f(x) = {\rm E} \left( f(x) \right) ^ 2 - \left( {\rm E} f(x) \right) ^ 2.
\end{equation}
%


\subsection{}
By the definition,
%
\begin{equation}
{\rm cov} (x, y) = {\rm E} xy - {\rm E} x {\rm E} y.
\end{equation}
%
Here,
%
\begin{equation}
{\rm E} x y = \int xy f(x, y) dx dy
\end{equation}
%
If $x$ and $y$ are independent, by the definition
%
\begin{equation}
f(x, y) = f(x) f(y),
\end{equation}
%
then
%
\begin{equation}
\int xy f(x, y) dx dy = \int f(x) dx \int f(y) dy.
\end{equation}
%
Therefore,
%
\begin{equation}
{\rm E} xy = {\rm E} x {\rm E} y,
\end{equation}
%
and thus
%
\begin{equation}
{\rm cov} (x, y) = 0.
\end{equation}
%


