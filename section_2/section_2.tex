\section{Probability Distributions}


\subsection{}
\label{2.1}
Let $x$ be a variable such that
%
\begin{equation}
p(x | \mu) = \mu ^ x (1 - \mu) ^ {1 - x},
\end{equation}
%
where $x \in \{ 0, 1 \}$.
Then,
%
\begin{equation}
\sum_{x} p(x | \mu) = 1.
\end{equation}
%
By the definition,
%
\begin{equation}
\begin{aligned}
{\rm E} x &= \mu, \\
{\rm E} x ^ 2 &= \mu,
\end{aligned}
\end{equation}
%
Since
%
\begin{equation}
{\rm var} x = {\rm E} x ^ 2 - \left( {\rm E} x \right) ^ 2,
\end{equation}
%
we have
%
\begin{equation}
{\rm var} x = \mu (1 - \mu).
\end{equation}
%
By the definition,
%
\begin{equation}
{\rm H} (x) = - \sum_{x} p(x | \mu) \ln p(x | \mu).
\end{equation}
%
Therefore,
%
\begin{equation}
{\rm H} (x) = - \mu \ln \mu - (1 - \mu) \ln (1 - \mu).
\end{equation}
%


\subsection{}
\label{2.2}
Let $x$ be a variable such that
%
\begin{equation}
p(x | \mu) = \left( \frac{1 - \mu}{2} \right) ^ \frac{1 - x}{2} \left( \frac{1 + \mu}{2} \right) ^ \frac{1 + x}{2},
\end{equation}
%
where $x \in \{ -1, 1 \}$.
Then,
%
\begin{equation}
\sum_{x} p(x | \mu) = 1.
\end{equation}
%
By the definition,
%
\begin{equation}
\begin{aligned}
{\rm E} x &= \mu, \\
{\rm E} x ^ 2 &= 1,
\end{aligned}
\end{equation}
%
Since
%
\begin{equation}
{\rm var} x = {\rm E} x ^ 2 - \left( {\rm E} x \right) ^ 2,
\end{equation}
%
we have
%
\begin{equation}
{\rm var} x = 1 - \mu ^ 2.
\end{equation}
%
By the definition,
%
\begin{equation}
{\rm H} (x) = - \sum_{x} p(x | \mu) \ln p(x | \mu).
\end{equation}
%
Therefore,
%
\begin{equation}
{\rm H} (x) = - \frac{1 - \mu}{2} \ln \frac{1 - \mu}{2} - \frac{1 + \mu}{2} \ln \frac{1 + \mu}{2}.
\end{equation}
%


\subsection{}
\label{2.3}
By the definition,
%
\begin{equation}
\begin{aligned}
{N \choose m} &= \frac{N!}{m! (N - m)!}, \\
{N \choose m - 1} &= \frac{N!}{(m - 1)! (N - m + 1)!}
\end{aligned}
\end{equation}
%
Therefore,
%
\begin{equation}
{N \choose m} + {N \choose m - 1} = \frac{(N - m + 1) N! + m N!}{m! (N - m + 1)!}.
\end{equation}
%
By the definition, the right hand side can be written as
%
\begin{equation}
\frac{(N + 1)!}{m! (N + 1 - m)!} = {N + 1 \choose m}.
\end{equation}
%
Thus,
%
\begin{equation}
{N \choose m} + {N \choose m - 1} = {N + 1 \choose m}.
\end{equation}
%

Note that 
%
\begin{equation}
1 + x = \sum_{m = 0}^{1} {1 \choose m} x ^ m.
\end{equation}
%
Let us assume that 
%
\begin{equation}
(1 + x) ^ N = \sum_{m = 0}^{N} {N \choose m} x ^ m.
\end{equation}
%
Then,
%
\begin{equation}
(1 + x) ^ {N + 1} = \sum_{m = 0}^{N} {N \choose m} x ^ m + \sum_{m = 0}^{N} {N \choose m} x ^ {m + 1}.
\end{equation}
%
By the result above, the right hand side can be written as
%
\begin{equation}
\sum_{m = 0}^{N} {N \choose m} x ^ m + \sum_{m = 1}^{N + 1} {N \choose m - 1} x ^ m = 1 + x ^ {N + 1} + \sum_{m = 1}^{N} {N + 1 \choose m} x ^ m.
\end{equation}
%
Therefore,
%
\begin{equation}
(1 + x) ^ {N + 1} = \sum_{m = 0}^{N + 1} {N + 1 \choose m} x ^ m.
\end{equation}
%
Thus, the assumption is proved by induction on $N$.

Finally, let $m$ be a variable such that
%
\begin{equation}
p(m | \mu) = {N \choose m} \mu ^ m (1 - \mu) ^ {N - m}.
\end{equation}
%
Then
%
\begin{equation}
\sum_{m = 0}^{N} p(m | \mu) = \sum_{m = 0}^{N} {N \choose m} \mu ^ m (1 - \mu) ^ {N - m}.
\end{equation}
%
By the result above, the right hand side can be written as
%
\begin{equation}
(1 - \mu) ^ N \sum_{m = 0}^{N} {N \choose m} \left( \frac{\mu}{1 - \mu} \right) ^ m = (1 - \mu) ^ N \left( 1 + \frac{\mu}{1 - \mu} \right) ^ N.
\end{equation}
%
Therefore,
%
\begin{equation}
\sum_{m = 0}^{N} p(m | \mu) = 1.
\end{equation}
%


\subsection{}
\label{2.4}
Let $m$ be a variable such that
%
\begin{equation}
p(m | \mu) = {N \choose m} \mu ^ m (1 - \mu) ^ {N - m}.
\end{equation}
%
Then
%
\begin{equation}
{\rm E} m = \sum_{m = 0}^{N} m {N \choose m} \mu ^ m (1 - \mu) ^ {N - m}.
\end{equation}
%
Differentiating both sides of 
%
\begin{equation}
\sum_{m = 0}^{N} {N \choose m} \mu ^ m (1 - \mu) ^ {N - m}  = 1
\end{equation}
%
with respect to $\mu$ gives
%
\begin{equation}
\sum_{m = 0}^{N} m {N \choose m} \mu ^ {m - 1} (1 - \mu) ^ {N - m} - \sum_{m = 0}^{N} (N - m) {N \choose m} \mu ^ m (1 - \mu) ^ {N - m - 1} = 0.
\end{equation}
%
The first term of the left hand side can be written as $\frac{1}{\mu} {\rm E} m$.
Since
%
\begin{equation}
(N - m) {N \choose m} = N {N - 1 \choose m},
\end{equation}
%
the second term of the left hand side can be written as
%
\begin{equation}
- N \sum_{m = 0}^{N - 1} {N - 1 \choose m} \mu ^ m (1 - \mu) ^ {N - m - 1} = - N.
\end{equation}
%
Therefore,
%
\begin{equation}
{\rm E} m = N \mu. 
\end{equation}
%

Differentiating both sides of 
%
\begin{equation}
\sum_{m = 0}^{N} {N \choose m} \mu ^ m (1 - \mu) ^ {N - m}  = 1
\end{equation}
%
twice with respect to $\mu$ gives
%
\begin{equation}
\begin{aligned}
&\sum_{m = 0}^{N} m (m - 1) {N \choose m} \mu ^ {m - 2} (1 - \mu) ^ {N - m} \\
&- 2 \sum_{m = 0}^{N} m (N - m) {N \choose m} \mu ^ {m - 1} (1 - \mu) ^ {N - m - 1} \\
&+ \sum_{m = 0}^{N} (N - m) (N - m - 1) {N \choose m} \mu ^ m (1 - \mu) ^ {N - m - 2} = 0.
\end{aligned}
\end{equation}
%
The first term of the left hand side can be written as $\frac{1}{\mu ^ 2} {\rm E} m (m - 1)$.
Since
%
\begin{equation}
\begin{aligned}
m (N - m) {N \choose m} &= N (N - 1) {N - 2 \choose m - 1}, \\
(N - m) (N - m - 1) {N \choose m} &= N (N - 1) {N - 2 \choose m},
\end{aligned}
\end{equation}
%
the second and third term of the left hand side can be written as
%
\begin{equation}
\begin{aligned}
- 2 N (N - 1) \sum_{m = 1}^{N - 1} {N - 2 \choose m - 1} \mu ^ {m - 1} (1 - \mu) ^ {N - m - 1} &= - 2 N (N - 1), \\
N (N - 1) \sum_{m = 0}^{N} {N - 2 \choose m} \mu ^ m (1 - \mu) ^ {N - m - 2} &= N (N - 1).
\end{aligned}
\end{equation}
%
Therefore,
%
\begin{equation}
{\rm E} m (m - 1) = N (N - 1) \mu ^ 2.
\end{equation}
%
Thus, since 
%
\begin{equation}
{\rm var} m = {\rm E} m (m - 1) + {\rm E} m - ( {\rm E} m ) ^ 2,
\end{equation}
%
we have
%
\begin{equation}
{\rm var} m = N \mu ( 1 - \mu).
\end{equation}
%


\subsection{}
\label{2.5}
By the definition,
%
\begin{equation}
\Gamma (a) \Gamma (b) = \int_{0}^{\infty} x ^ {a - 1} \exp (- x) dx \int_{0}^{\infty} y ^ {b - 1} \exp (- y) dy.
\end{equation}
%
By the transformation $t = x + y$, the right hand side can be written as
%
\begin{equation}
\begin{aligned}
&\int_{0}^{\infty} x ^ {a - 1} \left( \int_{x}^{\infty} (t - x) ^ {b - 1} \exp (- t) dt \right) dx \\
&=  \int_{0}^{\infty} \left( \int_{0}^{t} x ^ {a - 1} (t - x) ^ {b - 1} dx \right) \exp (- t) dt.
\end{aligned}
\end{equation}
%
By the transformation $x = t \mu$, the right hand side can be written as
%
\begin{equation}
\begin{aligned}
&\int_{0}^{\infty} \left( \int_{0}^{1} (t \mu) ^ {a - 1} t ^ {b - 1} (1 - \mu) ^ {b - 1} t d\mu \right) \exp (- t) dt \\
&= \int_{0}^{1} \mu ^ {a - 1} (1 - \mu) ^ {b - 1} d\mu \int_{0}^{\infty} t ^ {a + b - 1} \exp ( - t ) dt.
\end{aligned}
\end{equation}
%
By the definition, the second integral of the right hand side  can be written as $\Gamma(a + b)$.
%
Therefore,
%
\begin{equation}
\int_{0}^{1} \mu ^ {a - 1} (1 - \mu) ^ {b - 1} d\mu = \frac{\Gamma (a) \Gamma (b)}{\Gamma (a + b)}.
\end{equation}
%


\subsection{}
\label{2.6}
Let $\mu$ be a variable such that
%
\begin{equation}
p(\mu | a, b) = \frac{\Gamma(a + b)}{\Gamma(a) \Gamma(b)} \mu ^ {a - 1} (1 - \mu) ^ {b - 1}.
\end{equation}
%
Then
%
\begin{equation}
\begin{aligned}
{\rm E} \mu &= \frac{\Gamma(a + b)}{\Gamma(a) \Gamma(b)} \int_{0}^{1} \mu ^ a (1 - \mu) ^ {b - 1} d\mu, \\
{\rm E} \mu ^ 2 &= \frac{\Gamma(a + b)}{\Gamma(a) \Gamma(b)} \int_{0}^{1} \mu ^ {a + 1} (1 - \mu) ^ {b - 1} d\mu
\end{aligned}
\end{equation}
%
Since
%
\begin{equation}
\begin{aligned}
\int_{0}^{1} \mu ^ a (1 - \mu) ^ {b - 1} d\mu &= \frac{\Gamma (a + 1) \Gamma (b)}{\Gamma (a + b + 1)}, \\
\int_{0}^{1} \mu ^ {a + 1} (1 - \mu) ^ {b - 1} d\mu &= \frac{\Gamma (a + 2) \Gamma (b)}{\Gamma (a + b + 2)},
\end{aligned}
\end{equation}
%
we have
%
\begin{equation}
\begin{aligned}
{\rm E} \mu &= \frac{a}{a + b}, \\
{\rm E} \mu ^ 2 &= \frac{a (a + 1)}{(a + b) (a + b + 1)}.
\end{aligned}
\end{equation}
%
Since
%
\begin{equation}
{\rm var} \mu = {\rm E} \mu ^ 2 - \left( {\rm E \mu} \right) ^ 2,
\end{equation}
%
we have
%
\begin{equation}
{\rm var} \mu = \frac{a b}{(a + b) ^ 2 (a + b + 1)}.
\end{equation}
%
Since
%
\begin{equation}
\frac{\partial}{\partial \mu} p( \mu | a, b) = \frac{\Gamma(a + b)}{\Gamma(a) \Gamma(b)} \mu ^ {a - 1} (1 - \mu) ^ {b - 1} \left( \frac{a - 1}{\mu} - \frac{b - 1}{1 - \mu} \right),
\end{equation}
%
we have
%
\begin{equation}
{\rm mode} \mu = \frac{a - 1}{a + b - 2}.
\end{equation}
%









































