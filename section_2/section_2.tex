\section{Probability Distributions}


\subsection{}
\label{2.1}
Let $x$ be a variable such that
%
\begin{equation}
p(x | \mu) = \mu ^ x (1 - \mu) ^ {1 - x},
\end{equation}
%
where $x \in \{ 0, 1 \}$.
Then,
%
\begin{equation}
\sum_{x} p(x | \mu) = 1.
\end{equation}
%
By the definition,
%
\begin{equation}
\begin{aligned}
{\rm E} x &= \mu, \\
{\rm E} x ^ 2 &= \mu,
\end{aligned}
\end{equation}
%
Since
%
\begin{equation}
{\rm var} x = {\rm E} x ^ 2 - \left( {\rm E} x \right) ^ 2,
\end{equation}
%
we have
%
\begin{equation}
{\rm var} x = \mu (1 - \mu).
\end{equation}
%
By the definition,
%
\begin{equation}
{\rm H} (x) = - \sum_{x} p(x | \mu) \ln p(x | \mu).
\end{equation}
%
Therefore,
%
\begin{equation}
{\rm H} (x) = - \mu \ln \mu - (1 - \mu) \ln (1 - \mu).
\end{equation}
%


\subsection{}
\label{2.2}
Let $x$ be a variable such that
%
\begin{equation}
p(x | \mu) = \left( \frac{1 - \mu}{2} \right) ^ \frac{1 - x}{2} \left( \frac{1 + \mu}{2} \right) ^ \frac{1 + x}{2},
\end{equation}
%
where $x \in \{ -1, 1 \}$.
Then,
%
\begin{equation}
\sum_{x} p(x | \mu) = 1.
\end{equation}
%
By the definition,
%
\begin{equation}
\begin{aligned}
{\rm E} x &= \mu, \\
{\rm E} x ^ 2 &= 1,
\end{aligned}
\end{equation}
%
Since
%
\begin{equation}
{\rm var} x = {\rm E} x ^ 2 - \left( {\rm E} x \right) ^ 2,
\end{equation}
%
we have
%
\begin{equation}
{\rm var} x = 1 - \mu ^ 2.
\end{equation}
%
By the definition,
%
\begin{equation}
{\rm H} (x) = - \sum_{x} p(x | \mu) \ln p(x | \mu).
\end{equation}
%
Therefore,
%
\begin{equation}
{\rm H} (x) = - \frac{1 - \mu}{2} \ln \frac{1 - \mu}{2} - \frac{1 + \mu}{2} \ln \frac{1 + \mu}{2}.
\end{equation}
%


\subsection{}
\label{2.3}
By the definition,
%
\begin{equation}
\begin{aligned}
{N \choose m} &= \frac{N!}{m! (N - m)!}, \\
{N \choose m - 1} &= \frac{N!}{(m - 1)! (N - m + 1)!}
\end{aligned}
\end{equation}
%
Therefore,
%
\begin{equation}
{N \choose m} + {N \choose m - 1} = \frac{(N - m + 1) N! + m N!}{m! (N - m + 1)!}.
\end{equation}
%
By the definition, the right hand side can be written as
%
\begin{equation}
\frac{(N + 1)!}{m! (N + 1 - m)!} = {N + 1 \choose m}.
\end{equation}
%
Thus,
%
\begin{equation}
{N \choose m} + {N \choose m - 1} = {N + 1 \choose m}.
\end{equation}
%

Note that 
%
\begin{equation}
1 + x = \sum_{m = 0}^{1} {1 \choose m} x ^ m.
\end{equation}
%
Let us assume that 
%
\begin{equation}
(1 + x) ^ N = \sum_{m = 0}^{N} {N \choose m} x ^ m.
\end{equation}
%
Then,
%
\begin{equation}
(1 + x) ^ {N + 1} = (1 + x) \sum_{m = 0}^{N} {N \choose m} x ^ m.
\end{equation}
%
By the result above, the right hand side can be written as
%
\begin{equation}
\sum_{m = 0}^{N} \left( {N + 1 \choose m} - {N \choose m - 1} \right)  \left( x ^ m + x ^ {m + 1} \right)
\end{equation}
%



































